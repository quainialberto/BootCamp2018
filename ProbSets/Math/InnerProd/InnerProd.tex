\documentclass[11.5pt, letterpaper, bibtotoc,
    tablecaptionabove, figurecaptionabove]{article}


\setlength{\headheight}{10pt}
\setlength{\headsep}{15pt}
\setlength{\topmargin}{-25pt}
\setlength{\topskip}{0in}
\setlength{\textheight}{8.7in}
\setlength{\footskip}{0.3in}
\setlength{\oddsidemargin}{0.0in}
\setlength{\evensidemargin}{0.0in}
\setlength{\textwidth}{6.5in}

\usepackage{setspace}
\setstretch{1.2}
\setlength{\parskip}{5pt}%{6pt}
\setlength{\parindent}{0pt}

\usepackage{subfigure}
\usepackage{subfiles}
\usepackage{graphicx}
\usepackage{epsfig}
\graphicspath{{images/}{../images/}}
\usepackage{amsmath}
\usepackage{amssymb}
\usepackage{amsthm}
\usepackage{enumerate}
\newtheorem{proposition}{Proposition}
\newtheorem{remark}{Remark}
\newtheorem{lemma}{Lemma}
\newtheorem{notation}{Notation}
\newtheorem{corollary}{Corollary}
\newtheorem{remarks}{Remarks}
\newtheorem{examples}{Examples}
\newtheorem{assumption}{Assumption}
\newtheorem{definition}{Definition}
\newcommand{\norm}[1]{\left\lVert#1\right\rVert}
\DeclareMathOperator{\dom}{dom}
\DeclareMathOperator{\ri}{ri}
\DeclareMathOperator{\interior}{int}
\DeclareMathOperator{\essential}{ess}
\DeclareMathOperator{\range}{range}
\DeclareMathOperator{\diag}{diag}
\DeclareMathOperator{\rank}{rank}

\makeatletter
\newcommand{\leqnomode}{\tagsleft@true\let\veqno\@@leqno}
\newcommand{\reqnomode}{\tagsleft@false\let\veqno\@@eqno}
\newcommand{\vertiii}[1]{{\left\vert\kern-0.25ex\left\vert\kern-0.25ex\left\vert#1
\right\vert\kern-0.25ex\right\vert\kern-0.25ex\right\vert}}
\makeatother

\usepackage{bm}

\usepackage[utf8]{inputenc}
\usepackage[english]{babel}
\usepackage{hyperref}
\hypersetup{
    colorlinks=true,
    linkcolor=blue,
    citecolor=red,
}
\usepackage[margin=1in]{geometry}

\begin{document}

\textbf{Alberto Quaini}

\section*{Inner Product Spaces Exercises}

\subsection{Exercise 1}
(i)
\begin{align*}
    &\left( ||x+y||^2 - ||x-y||^2 \right)/4=\\
    &\left( <x,x> + <y,y> + 2<x,y> - <x,x> - <y,y> + 2<x,y>\right)/4=\\
    &<x,y>.
\end{align*}
(ii)
\begin{align*}
    &\left( ||x+y||^2 + ||x-y||^2 \right) / 4 =\\
    &\left( <x,x> + <y,y> + 2<x,y> + <x,x> + <y,y> - 2<x,y>\right) / 2 =\\
    &<x,x> + <y,y>.
\end{align*}

\subsection{Exercise 2}
\begin{align*}
    &(||x+y||^2 - ||x-y||^2 + i||x-iy||^2 - i||x+iy||^2) / 4 =\\
    &(<x+y,x+y> - <x-y,x-y> + i<x-iy,y-iy> - i<x+iy,x+iy>) / 4 =\\ 
    &(2<x,y> + 2<y,x> -2<x,y> +2<y,x>) / 4=\\
    &<x,y>.
\end{align*}

\subsection{Exercise 3}
$<x,x^5> = \int_0^1x^6dx=x^7/7|_0^1=1/7$,
$||x|| = \int_0^1x^2dx=x^3/3|_0^1=1/3$ and
$||x^5|| = \int_0^1x^10dx=x^11/11|_0^1=1/11$.
Therefore $\cos\theta=\sqrt{33}/7$ implies $\theta=34.5$.

\subsection{Exercise 4}
(i)
\begin{align*}
    ||\cos(t)||=\frac{1}{\pi}\int_{-\pi}^\pi\cos^2(t)dt=
    \frac{1}{\pi}\left.\frac{\cos(x)\sin(x)-x}{2}\right\lvert_{-\pi}^pi=\frac{\pi}{\pi}=1,
\end{align*}
and similarly $||\sin(t)||=1$.
Also
\begin{align*}
    ||\cos(2t)||=\frac{1}{\pi}\int_{-\pi}^\pi\cos^2(2t)dt=
    \frac{1}{\pi}\left.\frac{\sin(4t)+4t}{8}\right\lvert_{-\pi}^pi=\frac{\pi}{\pi}=1,
\end{align*}
and similarly $||\sin(2t)||=1$.
Therefore the basis is normalized.

The following integrals:
\begin{align*}
    <\cos(t),\sin(t)> = \frac{1}{\pi}\int_{-\pi}^\pi\cos(t)\sin(t)dt=
    \frac{1}{\pi}\left.\frac{\sin^2(x)}{x}\right\lvert_{-\pi}^pi=0,
\end{align*}

\begin{align*}
    <\cos(t),\cos(2t)> = \frac{1}{\pi}\int_{-\pi}^\pi\cos(t)\cos(2t)dt=
    \frac{1}{\pi}\left.frac{3\sin(t)-2\sin^3(t)}{3}\right\lvert_{-\pi}^pi=0,
\end{align*}

\begin{align*}
    <\cos(t),\sin(2t)> = \frac{1}{\pi}\int_{-\pi}^\pi\cos(t)\sin(2t)dt=
    \frac{1}{\pi}\left.frac{-2\cos^3(t)}{3}\right\lvert_{-\pi}^pi=0,
\end{align*}

\begin{align*}
    <\cos(2t),\sin(2t)> = \frac{1}{\pi}\int_{-\pi}^\pi\cos(2t)\sin(2t)dt=
    \frac{1}{\pi}\left.frac{-\cos^2(2t)}{4}\right\lvert_{-\pi}^pi=0,
\end{align*}

and so on, shows that $S$ is an orthonormal basis.

(ii)
\begin{align*}
    ||t|| = \frac{1}{\pi}\int_{-\pi}^\pi tdt=
    \frac{1}{\pi}\left.\frac{t^2}{2}\right\lvert_{-\pi}^\pi=0.
\end{align*}

\end{document}
