\documentclass[11.5pt, letterpaper, bibtotoc,
    tablecaptionabove, figurecaptionabove]{article}


\setlength{\headheight}{10pt}
\setlength{\headsep}{15pt}
\setlength{\topmargin}{-25pt}
\setlength{\topskip}{0in}
\setlength{\textheight}{8.7in}
\setlength{\footskip}{0.3in}
\setlength{\oddsidemargin}{0.0in}
\setlength{\evensidemargin}{0.0in}
\setlength{\textwidth}{6.5in}

\usepackage{setspace}
\setstretch{1.2}
\setlength{\parskip}{5pt}%{6pt}
\setlength{\parindent}{0pt}

\usepackage{subfigure}
\usepackage{subfiles}
\usepackage{graphicx}
\usepackage{epsfig}
\graphicspath{{images/}{../images/}}
\usepackage{amsmath}
\usepackage{amssymb}
\usepackage{amsthm}
\usepackage{enumerate}
\newtheorem{proposition}{Proposition}
\newtheorem{remark}{Remark}
\newtheorem{lemma}{Lemma}
\newtheorem{notation}{Notation}
\newtheorem{corollary}{Corollary}
\newtheorem{remarks}{Remarks}
\newtheorem{examples}{Examples}
\newtheorem{assumption}{Assumption}
\newtheorem{definition}{Definition}
\newcommand{\norm}[1]{\left\lVert#1\right\rVert}
\DeclareMathOperator{\dom}{dom}
\DeclareMathOperator{\ri}{ri}
\DeclareMathOperator{\interior}{int}
\DeclareMathOperator{\essential}{ess}
\DeclareMathOperator{\range}{range}
\DeclareMathOperator{\diag}{diag}
\DeclareMathOperator{\rank}{rank}

\makeatletter
\newcommand{\leqnomode}{\tagsleft@true\let\veqno\@@leqno}
\newcommand{\reqnomode}{\tagsleft@false\let\veqno\@@eqno}
\newcommand{\vertiii}[1]{{\left\vert\kern-0.25ex\left\vert\kern-0.25ex\left\vert#1
\right\vert\kern-0.25ex\right\vert\kern-0.25ex\right\vert}}
\makeatother

\usepackage{bm}

\usepackage[utf8]{inputenc}
\usepackage[english]{babel}
\usepackage{hyperref}
\hypersetup{
    colorlinks=true,
    linkcolor=blue,
    citecolor=red,
}
\usepackage[margin=1in]{geometry}

\begin{document}

\textbf{Alberto Quaini}

\textbf{\Large Measure theory exercises}

\textbf{Exercise 1.3}
\begin{enumerate}
\item
$\mathcal G_1 := \{A: A\subset\mathbb R, A \text{ is open}\}$ is not an algebra on $\mathbb R$, hence not a $\sigma-$algebra.
Let $a\in\mathbb R$ and define $A_1:=(-\infty, a)$. 
Clearly $A_1\in\mathcal G_1$.
However, $A_1^c = [a, +\infty)\notin\mathcal G_1$.

\item
$\mathcal G_2 := \{A: A \text{ is a finite union of intervals of the form } (a, b], (-\infty, b], (a, \infty)\}$ is an algebra, but not a $\sigma-$algebra.
take $a,b\in\mathbb R$ such that $a = b$, then $(a, b]=\emptyset$ and $(a, b]\in\mathcal G_2$.
Also, $\mathcal G_2$ is closed under complements.
The former is true because $(a, b]^c=(-\infty, a]\cap(b, \infty)\in\mathcal G_2$,
$(-\infty, b]^c=(b, \infty)\in\mathcal G_2$ and $(a, \infty)^c=(-\infty, a]\in\mathcal G_2$.
We can also show that the complement of any finite union of intervals of the the form
$(a, b]$, $(-\infty, b]$ and $(a, \infty)$ is also in $\mathcal G_2$ by the properties of complements.
Finally, $\mathcal G_2$ is closed under finite union as the finite union of finite unions of intervals of the form $(a, b]$, $(-\infty, b]$ and $(a, \infty)$ is still a finite union.
However, $\mathcal G_2$ is not a $\sigma-$algebra as an infinite union of finite unions of intervals of the form $(a, b]$, $(-\infty, b]$ and $(a, \infty)$ is an infinite union of intervals of the form $(a, b]$, $(-\infty, b]$ and $(a, \infty)$ and thus it would not belong to $\mathcal G_2$.

\item
$\mathcal G_3 := \{A: A \text{ is a countable union of intervals of the form } (a, b], (-\infty, b], (a, \infty)\}$ is a $\sigma-$algebra, hence also an algebra.
Everything discussed for $\mathcal G_2$ holds except for the fact that infinite unions of the basic intervals belong to $\mathcal G_3$. 

\end{enumerate}

\textbf{Exercise 1.7}
By definition, any $\sigma-$algebra contains $\emptyset$.
Also, it contains $X$, since it must be closed under complements.
Therefore, $\{\emptyset, X\}$ is contained in any $\sigma-$algebra and is thus the smallest $\sigma-$algebra on $X$.

On the other hand, by definition, any $\sigma-$algebra on $X$ is a set of subsets of $X$, 
therefore it is contained in the power set of $X$, which is the set of all subsets of $X$.

\textbf{Exercise 1.10}
Let $\mathcal N = \underset{\alpha}{\cap}\mathcal S_\alpha$.
$\emptyset\in\mathcal N$ because $\emptyset\in\mathcal S_\alpha$, for every $\alpha$.
Also, if $A\in\mathcal N$, we have that $A$ is in every $\mathcal S_\alpha$,
and since these are closed under complements, $A^c\in\mathcal N$.
Finally, if $A_1, A_2,\ldots\in\mathcal N$, they belong to each $\mathcal S_\alpha$ and so does 
$\cup_{n=1}^\infty A_n$, and we have that it also belongs to $\mathcal N$.
In conclusion, $\mathcal N$ is a $\sigma-$algebra.

\textbf{Exercise 1.17}
\begin{enumerate}
\item
Take $A,B\in\mathcal S$, with $A\subset B$.
Notice that $B$ can be written as the union of two disjoint sets in the following way: 
$B = (A\cap B) \cup (A^c\cap B)$.
Then, $\mu(B)=\mu(A\cap B) + \mu(A^c\cap B) = \mu(A) + \mu(A^c\cap B)$.
Since a measure is nonnegative, $\mu(B)\geq\mu(A)$.
Therefore $\mu$ is monotone.

\item
Let $\{A_n\}_{n\in\mathbb N}\subset\mathcal S$ and define the following sets: $A := \cup_{n\in\mathbb N} A_n$,
$B_1 := A_1$, $B_2 := A_2-A_1$, $B_3 := A_3-(A_1\cup A_2)$, and so on.
Then, $A = \cup_{n\in\mathbb N} B_n$.
By monotonicity, for each $n\in\mathbb N$, $\mu(B_n)\leq\mu(A_n)$ since $B_n\subset A_n$.
Therefore we obtain $\mu(A)=\sum_{n\in\mathbb N}\mu(B_n)\leq \sum_{n\in\mathbb N}\mu(A_n)$.
\end{enumerate}

\textbf{Exercise 1.18}
$\lambda$ is a measure because (i) $\lambda(\emptyset) = \mu(\emptyset\cap B) = \mu(\emptyset) = 0$
and (ii) for any $\{A_n\}_{n\in\mathbb N}\subset\mathcal S$ with $A_n$'s paiwise disjoint we have
$\lambda(\cup_{n\in\mathbb N} A_n) = \mu\left((\cup_{n\in\mathbb N} A_n) \cap B\right) =
\mu\left(\cup_{n\in\mathbb N} (A_n \cap B)\right) = \sum_{n\in\mathbb N}\mu(A_n\cap B) = \sum_{n\in\mathbb N}\lambda(A_n)$.

\textbf{Exercise 1.20}
Since $\mu(A_1)<\infty$, by monotonicity $\mu(A_i)<\infty$ for each $n\in\mathbb N$.
Consider the increasing sequence $\{A_1-A_n\}_{n\in\mathbb N}$,
define $A=\cap_{n\in\mathbb N} A_n$
and note that $\lim_{n\to\infty}(A_1-A_n) = A_1-\lim_{n\to\infty}A_n = A_1-A$.
Since $\mu$ is continuous from below,
\begin{equation*}
\mu(A_1)-\mu(A_n) = \mu(A_1-A_n) \underset{n\to\infty}{\longrightarrow}
\mu(A_1-A) = \mu(A_1)-\mu(A).
\end{equation*}
Therefore $\mu(A)=\lim_{n\to\infty}\mu(A_n)$.

\end{document}