\documentclass[11.5pt, letterpaper, bibtotoc,
    tablecaptionabove, figurecaptionabove]{article}


\setlength{\headheight}{10pt}
\setlength{\headsep}{15pt}
\setlength{\topmargin}{-25pt}
\setlength{\topskip}{0in}
\setlength{\textheight}{8.7in}
\setlength{\footskip}{0.3in}
\setlength{\oddsidemargin}{0.0in}
\setlength{\evensidemargin}{0.0in}
\setlength{\textwidth}{6.5in}

\usepackage{setspace}
\setstretch{1.2}
\setlength{\parskip}{5pt}%{6pt}
\setlength{\parindent}{0pt}

\usepackage{subfigure}
\usepackage{subfiles}
\usepackage{graphicx}
\usepackage{epsfig}
\graphicspath{{images/}{../images/}}
\usepackage{amsmath}
\usepackage{amssymb}
\usepackage{amsthm}
\usepackage{enumerate}
\newtheorem{proposition}{Proposition}
\newtheorem{remark}{Remark}
\newtheorem{lemma}{Lemma}
\newtheorem{notation}{Notation}
\newtheorem{corollary}{Corollary}
\newtheorem{remarks}{Remarks}
\newtheorem{examples}{Examples}
\newtheorem{assumption}{Assumption}
\newtheorem{definition}{Definition}
\newcommand{\norm}[1]{\left\lVert#1\right\rVert}
\DeclareMathOperator{\dom}{dom}
\DeclareMathOperator{\ri}{ri}
\DeclareMathOperator{\interior}{int}
\DeclareMathOperator{\essential}{ess}
\DeclareMathOperator{\range}{range}
\DeclareMathOperator{\diag}{diag}
\DeclareMathOperator{\rank}{rank}

\makeatletter
\newcommand{\leqnomode}{\tagsleft@true\let\veqno\@@leqno}
\newcommand{\reqnomode}{\tagsleft@false\let\veqno\@@eqno}
\newcommand{\vertiii}[1]{{\left\vert\kern-0.25ex\left\vert\kern-0.25ex\left\vert#1
\right\vert\kern-0.25ex\right\vert\kern-0.25ex\right\vert}}
\makeatother

\usepackage{bm}

\usepackage[utf8]{inputenc}
\usepackage[english]{babel}
\usepackage{hyperref}
\hypersetup{
    colorlinks=true,
    linkcolor=blue,
    citecolor=red,
}
\usepackage[margin=1in]{geometry}

\begin{document}

\textbf{Alberto Quaini}

\subsection*{Exercise 3}
$D:V\to V$ can be represented in matrix form as
\begin{equation*}
    \begin{bmatrix}
        0 & 2 & 0\\
        0 & 0 & 1\\
        0 & 0 & 0
    \end{bmatrix},
\end{equation*}
which is upper triangular with all diagonal elements zero.
Thus, all eigenvalues are $0$, with algebraic multiplicity $3$.
However, if $x$ is an eigenvector of $D$ corresponding to $\lambda = 0$,
then $Dx=0$.
Given the form of $D$ we conclude that $x_2=x_3=0$ and so the eigenspace
of $\lambda=0$ is $\text{span}\{1\}$.
Therefore, $\lambda=0$ has geometric multiplicity $1$.

\subsection*{Exercise 4}
(i)
Let
\begin{equation*}
    A = 
    \begin{bmatrix}
        a & b\\
        c & d
    \end{bmatrix}.
\end{equation*}
Since $A$ is hermitian, we know that $a = \bar{a}$, $d = \bar{d}$ and $b = \bar{c}$.
Then 
\begin{align*}
    \text{det}A &= ad - cb = ad - c\bar{c} = ad - ||c||^2 =\\
    &\bar{a}\bar{d}-||c||^2 = \overline{ad - ||c||^2}=\overline{\text{det}A}
\end{align*}    
and 
\begin{align*}
    \text{Tr}(A) = a + d = \bar{a} + \bar{d} = \overline{a + d} = \overline{\text{Tr}(A)}.
\end{align*}
Thus both the determinant and the trace of $A$ are real.
Notice that usign Exercise $3$ we have 

\begin{align*}
    \lambda_{1,2} = \frac{(a + d) \pm \sqrt{(a + d)^2 - 4(ad-||c||^2)}}{2}
\end{align*}
and the discriminant becomes $(a-d)^2 + 4||c||^2$, which is real and nonnegative,
therefore $A$ has only real eigenvalues.

(ii)
If $A$ is skew-symmetric, then $a=-\bar{a}$ and $d=-\bar{d}$, so they are imaginary, and $b=-\bar{c}$.
Thus $bc=-||c||^2$, and 
\begin{align*}
    \lambda_{1,2} = \frac{(a + d) \pm \sqrt{(a - d)^2 - 4||c||^2}}{2}.
\end{align*}
Let $a=\alpha i$ and $d=\beta i$. Then $(a-d)^2=i^2(\alpha-\beta)^2$ is clearly negative.
Therefore the discriminant is negative and the eigenvalues are all imaginary.

\subsection*{Exercise 6}
Let $R\in\mathbb M_n(\mathbb F)$ be an upper-triangular matrix with diagonal entries $r_{ii}$. 
Then $\lambda I-R$ is also upper-triangular and so $\text{det}R = \Pi_{i=1}^n(\lambda_i-r_{ii})$.
Since $r_{ii}$ are the roots of the characteristic polynomials, $\lambda_i=r_{ii}$.

\subsection*{Exercise 8}
(i)
We know that $V$ is the span of $S$.
If the vectors in $S$ are linearly independent, then $S$ is a basis for $V$.
From Problem Set 2 we noticed that the vectors in $S$ are orthonormal under the inner product
$<a,b>:=\frac{1}{\pi}\int_{-\pi}^{\pi}a(x)b(x)dx$.
Therefore they are independent and are thus a basis of $V$.

(ii)
Since $d\sin(x)/dx=cos(x)$, $d\cos(x)/dx=-\sin(x)$, $d\sin(2x)/dx=2\cos(2x)$ and
$d\cos(2x)/dx=-2\sin(2x)$, we have that
\begin{align*}
    D = 
    \begin{bmatrix}
        0 & -1 & 0 & 0\\
        1 & 0 & 0 & 0\\
        0 & 0 & 0 & 2\\
        0 & 0 & -2 & 0
    \end{bmatrix}.
\end{align*}

(iii)
Two complementary $D-$invariant subspaces are $\text{span}\{\sin(x), \cos(x)\}$
and $\text{span}\{\sin(2x), \cos(2x)\}$.

\subsection*{Exercise 13}
Since $\text{det}(\lambda I-A)=\lambda^2-1.4\lambda+0.4$, the eigenvalues are $1$ and $0.4$,
with corresponding eigenvectors $(2, 1)$, and $(1, -1)$.
Therefore
\begin{align*}
    P =
    \begin{bmatrix}
        2 & 1\\
        1 & -1
    \end{bmatrix}.
\end{align*}

\subsection*{Exercise 15}
A is semisimple, thus there exist matrices $\Lambda$ and $P$ such that
$A = P\Lambda P^{-1}$.
Then 
\begin{align*}
    f(A) = a_0PP^{-1}+a_1P\Lambda P^{-1}+
    \ldots+a_nP\Lambda^nP^{-1}=
    Pf(\Lambda)P^{-1},
\end{align*}
where $f(\Lambda)$ is diagonal with elements $(f(\lambda_i))_{i=1}^n$.
Since $f(A)$ is similar to $f(\Lambda)$, they have the same eigenvalues.

\subsection*{Exercise 16}
(i)
By Proposition 4.3.10,
\begin{align*}
    A^k = 
    \frac{1}{3}
    \begin{bmatrix}
        2 & 1\\ 1 & -1
    \end{bmatrix}
    \begin{bmatrix}
        1 & 0\\ 0 & 0.4^k
    \end{bmatrix}
    \begin{bmatrix}
        2 & 1\\ 1 & -1
    \end{bmatrix}.
\end{align*}
Consider the matrix 
\begin{align*} 
    B = 
    \frac{1}{3}
    \begin{bmatrix}
        2 & 1\\ 1 & -1
    \end{bmatrix}
    \begin{bmatrix}
        1 & 0\\ 0 & 0
    \end{bmatrix}
    \begin{bmatrix}
        2 & 1\\ 1 & -1
    \end{bmatrix} =
    \frac{1}{3}
    \begin{bmatrix}
        2 & 2\\ 1 & 1
    \end{bmatrix}.
\end{align*}
Their difference is
\begin{align*}
    A^k - B =
    \frac{1}{3}
    \begin{bmatrix}
        0.4^k & -2\times0.4^k\\ -0.4^k & 2\times0.4^k
    \end{bmatrix},
\end{align*}
and its $1-$norm is $4/3\times0.4^k$, which converges to 0.

(ii)
The $\infty-$norm of $A^k-B$ is $0.4^k$, whereas the Frobenius norm is
\begin{align*}
    \sqrt{\text{tr}\left(\left(A^k-B\right)^T\left(A^k-B\right)\right)}=
    \sqrt{10\times0.4^{2k}}
\end{align*}
and both of them converge to zero.

(iii)
By theorem 4.3.12, the eigenvalues of $3I+5A+A^3$ are given by
$f(\lambda_i) = 3+5\lambda_i+\lambda_i^3$, where $\lambda_i$'s are the eigenvalues of $A$.
So the eigenvalues are $f(1) = 9$ and $f(0.4)=5.064$.

\subsection*{Exercise 18}
Let $\lambda$ be an eigenvalue of $A$, then it is also an eigenvalue of $A^T$.
Then there exists a nonzero vector $x$ such that $A^Tx=\lambda x$.
Transposing both the RHS and the LHS we get the desired result.

\subsection*{Exercise 20}
Since $A$ is orthonormally similar to $B$, we know that there exists an orthonormal $P$
such that $B = PAP^H$.
Since $A$ is hermitian,
\begin{align*}
    B^H = (PAP^H)^H = PA^HP^H = PAP^H = B.
\end{align*}

\subsection*{Exercise 24}
First notice that the denominator is real nonnegative.
Also, notice that if $A$ is hermitian, then
\begin{align*}
    x^HAx = x^HA^Hx = (x^HAx)^H = \overline{x^HAx}.
\end{align*}
Thus $<x,Ax>=\overline{<x,Ax>}$, and so it is real.
On the other hand, if $A$ is skew-hermitian, then
\begin{align*}
    x^HAx = -x^HA^Hx = -(x^HAx)^H = -\overline{x^HAx}.
\end{align*}
Thus $<x,Ax> = -\overline{<x,Ax>}$, and is therefore imaginary.

\subsection*{Exercise 25}
(i)
Take an arbitrary vector $x$ in $\mathbb C^n$, then there exist coefficients $a_i$'s
such that $x=\sum_ia_ix_i$, since $\{x_i\}_i$ is a basis.
Then
\begin{align*}
    \left(\sum_jx_jx_j^H\right)\sum_ia_ix_i =
    \sum_jx_jx_j^Ha_jx_j + \sum_j\sum_{i\neq j}x_jx_j^Ha_ix_i =
    \sum_ja_jx_j
\end{align*}
because $x_j^Hx_j=1$ for any $j$ and $x_j^Hx_i=0$ for any $i\neq j$.
Thus $(\sum_jx_jx_j^H)x=x$ for any $x$ in $\mathbb C^n$. 
It must then be that $\sum_jx_jx_j^H=I$.

(ii)
Notice that
\begin{align*}
    Ax = \sum_jAa_jx_j = \sum_ja_j\lambda_jx_j
\end{align*}
and
\begin{align*}
    \left(\sum_j\lambda_jx_jx_j^H\right)\left(\sum_ia_ix_i\right)=
    \sum_j\lambda_jx_jx_j^Ha_jx_j + \sum_j\sum_{i\neq j}\lambda_jx_jx_j^Ha_ix_i =
    \sum_ja_j\lambda_jx_j,
\end{align*}
shows that $A=\sum_j\lambda_jx_jx_j^H$.

\subsection*{Exercise 27}
Since $A$ is positive definite, it is hermitian, hence its diagonal elements are reals.
Also, let $e_i$ denote $i^\text{th}$ standard basis vector of $\mathbb F^n$.
Then we have that, for any $i$:
\begin{align*}
    a_{ii} = e_i^HAe_i = \langle e_i,Ae_i\rangle > 0.
\end{align*}

\subsection*{Exercise 28}

By proposition $4.5.7$, There exist matrices $S_A$ and $S_B$ 
such that $A=S_A^HA_A$ and $B = S_B^HS_B$.
Then 
\begin{align*}
    \text{Tr}(AB) = \text{Tr}(S_A^HS_AS_B^HS_B) =
    \text{Tr}(S_BS_A^HS_AS_B^H) = \text{Tr}((S_AS_B^H)^HS_AS_B^H)
    = ||S_AS_B^H||_F^2\geq 0.
\end{align*}

By Proposition $4.5.6$ $A=Q_AD_AQ_A^H$ and $B=Q_BD_BQ_D^H$, where $Q_A$ and $Q_B$
are orthonormal and $D_A$, $D_B$ are diagonal matrices containing the eigenvalues
of $A$ and $B$ respectively.
Since the trace is invariant under orthonormal transformations we have
\begin{align*}
    \text{Tr}(AB)=\text{Tr}(D_AD_B)=\sum_i\lambda_i^A\lambda_i^B\leq
    \left(\sum_i\lambda_i^A\right)\left(\sum_i\lambda_i^B\right)=\text{Tr}(A)\text{Tr}(B),
\end{align*}
which concludes the proof.

\subsection*{Exercise 31}
(i)
Let $B=A^HA$, then $B$ is hermitian.
Then by Corollary $4.4.8$ $B$ has an orthonormal eigenbasis, say $\{b_i\}_{i=1}^n$,
which spans $\mathbb F^n$,
and real eigenvalues $\{\sigma_i\}_{i=1}^n$.
Take an arbitrary $x\in\mathbb F^n$ and real $(a_i)_{i=1}^n$ such that
$x=\sum_ia_ib_i$.
We have 
\begin{align*}
    ||x||_2=<\sum_ia_ib_i,\sum_ia_ib_i>^{1/2}=
    \sqrt{\sum_ia_i^2}
\end{align*}
since the $b_i$'s are orthonormal.
Also
\begin{align*}
    Bx=B\left(\sum_ia_ib_i\right)=\sum_ia_i\sigma_ib_i.
\end{align*}

Let $\sigma_1$ be the largest eigenvalue of $B$.
Then
\begin{align*}
    ||Ax|| =& <Ax,Ax> = <x,A^HAx> = 
    <x,Bx> =\\ &<\sum_ia_ib_i,\sum_i\sigma_ia_ib_i> =
    \sqrt{\sum_ia_i\sigma_i\bar{a}_i} \leq
    ||x||\max_i\sqrt{|\sigma_i|}.
\end{align*}
So $||A||=\sup\{||Ax||\ :\ ||x||=1\}\leq\max_i\sqrt{|\sigma_i|}$.

If we pick $x=b_1$, then
\begin{align*}
    ||A||\geq <b_1,Bb_1> = <b_1,\sigma_1b_1> = \sqrt{|\sigma_1|},
\end{align*}
which proves the result.

(ii)
Let $A = U \Sigma V^H$ be the singular value decomposition of $A$. 
Then, $A^{-1} = (V^H)^{-1} \Sigma^{-1} U^{-1}$. 
This is also a singular value decomposition, since $(V^H)^{-1} = V$ and 
$U^{-1} = U^H$, because $U$ and $V$ are orthonormal matrices, and $\Sigma^{-1}$ is diagonal. 
The largest singular value of $\Sigma^{-1}$ is $\frac{1}{\sigma_n}$, where $\sigma_n$ is the smallest one of $A$. 
Therefore, by (i) we have that  $\norm{A^{-1}}_2 = \sigma^{-1}_n$.

(iii)
Let $A = U \Sigma V^H$. 
Then,
\begin{align*}
    A^H = V \Sigma^H U^H 
	A^T = \overline{V} \Sigma U^T 
	A^H A = (V \Sigma^H U^H) (U \Sigma V^H) = V (\Sigma^H \Sigma) V^H
\end{align*}
All of these are singular value decompositions. 
Now, consider the singular values of each of these decompositions (where $\Sigma^H = \Sigma$ because singular values are reals). 
Thus, $A^H, A^T$ and $A$ all have the same singular values.
So by (i) we have that $\norm{A^H}^2_2 = \norm{A^T}^2_2 =  \norm{A}^2_2 = \sigma^2_1$. 
Also, notice that the diagonal elements of $(\Sigma^H \Sigma)$ are the singular values squared. 
We conclude that $\norm{A^H A}_2  = \sigma^2_1$.

\subsection*{Exercise 36}
$-I$ has both eigenvalues $-1$ and both singular values $1$.

\subsection*{Exercise 38}
I will prove (1) and (5), the other proofs follow a similar logic.

(1)

Let $A = U_1 \Sigma_1 V_1^H$ be the compact SVD of $A$. 
The Moore-Penrose pseudoinverse of $A$ is $A^{\dagger} = V_1 \Sigma^{-1}_1 U^H_1$.
Notice that,
\begin{align*}
    A A^{\dagger} A  &= (U_1 \Sigma_1 V_1^H)(V_1 \Sigma^{-1}_1 U^H_1) (U_1 \Sigma_1 V_1^H)
	=  U_1 \Sigma_1 (V_1^HV_1) \Sigma^{-1}_1 (U^H_1 U_1) \Sigma_1 V_1^H \\
	&= U_1 \Sigma_1  \Sigma^{-1}_1 \Sigma_1 V_1^H
	= U_1 \Sigma_1 V_1^H = A
\end{align*}
Therefore, $ A A^{\dagger} A = A$.

(5)

First, notice that by (1), $AA^\dagger AA^\dagger = AA^\dagger$, so that $AA^\dagger$ is idempotent.
Then, notice that $AA^\dagger=U_1U_1^H$ where $U_1$ is an orthonormal basis for the range of $A$.
Let $u_i$ denote the $m-$ dimensional $i^{th}$ column of $U_1$. 
Then
\begin{align*}
    U_1U_1^Hx=U_1(u_1^Hx, \ldots, u_r^Hx)^T
    = \sum_iu_iu_i^Hx = \sum_iu_i^Hxu_i  = \sum_i<u_i,x>u_i,
\end{align*}
which is the projection of $x$ on the range of $A$.


\end{document}
