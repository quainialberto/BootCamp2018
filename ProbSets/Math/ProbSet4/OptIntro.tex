\documentclass[11.5pt, letterpaper, bibtotoc,
    tablecaptionabove, figurecaptionabove]{article}


\setlength{\headheight}{10pt}
\setlength{\headsep}{15pt}
\setlength{\topmargin}{-25pt}
\setlength{\topskip}{0in}
\setlength{\textheight}{8.7in}
\setlength{\footskip}{0.3in}
\setlength{\oddsidemargin}{0.0in}
\setlength{\evensidemargin}{0.0in}
\setlength{\textwidth}{6.5in}

\usepackage{setspace}
\setstretch{1.2}
\setlength{\parskip}{5pt}%{6pt}
\setlength{\parindent}{0pt}

\usepackage{subfigure}
\usepackage{subfiles}
\usepackage{graphicx}
\usepackage{epsfig}
\graphicspath{{images/}{../images/}}
\usepackage{amsmath}
\usepackage{amssymb}
\usepackage{amsthm}
\usepackage{enumerate}
\newtheorem{proposition}{Proposition}
\newtheorem{remark}{Remark}
\newtheorem{lemma}{Lemma}
\newtheorem{notation}{Notation}
\newtheorem{corollary}{Corollary}
\newtheorem{remarks}{Remarks}
\newtheorem{examples}{Examples}
\newtheorem{assumption}{Assumption}
\newtheorem{definition}{Definition}
\newcommand{\norm}[1]{\left\lVert#1\right\rVert}
\DeclareMathOperator{\dom}{dom}
\DeclareMathOperator{\ri}{ri}
\DeclareMathOperator{\interior}{int}
\DeclareMathOperator{\essential}{ess}
\DeclareMathOperator{\range}{range}
\DeclareMathOperator{\diag}{diag}
\DeclareMathOperator{\rank}{rank}

\makeatletter
\newcommand{\leqnomode}{\tagsleft@true\let\veqno\@@leqno}
\newcommand{\reqnomode}{\tagsleft@false\let\veqno\@@eqno}
\newcommand{\vertiii}[1]{{\left\vert\kern-0.25ex\left\vert\kern-0.25ex\left\vert#1
\right\vert\kern-0.25ex\right\vert\kern-0.25ex\right\vert}}
\makeatother

\usepackage{bm}

\usepackage[utf8]{inputenc}
\usepackage[english]{babel}
\usepackage{hyperref}
\hypersetup{
    colorlinks=true,
    linkcolor=blue,
    citecolor=red,
}
\usepackage[margin=1in]{geometry}

\begin{document}

\textbf{Alberto Quaini}

\subsection*{Exercise 6.1}
Let $f(w):=-e^{-w^Tx}$, $G(w):=w^TAw - w^TAy - w^Tx$,
and $H(w)=y^Tw-w^Tx$. 
Then the problem can be written in the following standard form:
\begin{align*}
    &\text{minimize}\quad\ f(w)\\
    &\text{subject to}\quad G(w)\leq a\\
    &\qquad \quad \quad \quad\ H(w)=b.
\end{align*}

\subsection*{Exercise 6.6}
The gradient is $Df(x,y)=(6xy+4y^2+y, 3x^2+8xy+x)$ and the hessian is
\begin{align*}
    D^2f(x,y) = 
    \begin{bmatrix}
        6y & 6x+8y+1\\ 6x+8y+1 & 8x
    \end{bmatrix}.
\end{align*}
The first order conditions read $Df(x,y)=(0,0)$ and yield the following critical points:
$A=(-1/3,0)$, $B=(-1/9, -1/12)$, $C=(0,0)$ and $D=(0,-1/4)$.
The eigenvalues of the hessian for $A$ are approximately $0.3$ and $-3$, thus $A$ is a saddle point.
The eigenvalues of the hessian for $B$ are approximately $-0.3$ and $-1.1$, thus $B$ is a local maximizer.
The eigenvalues of the hessian for $C$ are approximately $1$ and $-1.1$, thus $C$ is a saddle point.
The eigenvalues of the hessian for $D$ are approximately $-2$ and $0.5$, thus $D$ is a saddle point.

\subsection*{Exercise 6.7}
(i)

Notice that $Q^T = (A^T + A)^T = A^T+ A = A + A^T = Q$.
Also, $x^TAx = \sum_{i=1}^na_{ij}x_ix_j = \sum_{i=1}^na_{ji}x_ix_j = x^TA^Tx$.
Therefore $x^TQx = 2x^TAx$ and $(6.17)$ is equivalent to
\begin{align*}
    f(x) = x^TQx/2 - b^Tx +c.
\end{align*}

(ii)

The first order necessary conditions for a minimizer imply
$Q^Tx^* = b$, since $f'(x) = Q^Tx-b$.

(iii)

If $Q$ is positive definite, then $f''(x)>0$ for any $x$.
Also, $Q$ is invertible and by $(6.19)$ we have 
that $x^*=Q^{-1}b$ is such that $f'(x^*)=0$.
Then by the second order sufficient condition, $x^*$ is the unique minimizer of $f$.
Now assume $x^*$ is the unique minimizer of $f$.
Then by the second order necessary condition, $Q$ is positive semi-definite.
Also, $x^*$ is a solution to $Q^Tx^*=b$.
If $Q$ has at least one zero eigenvalue, then $x^*$ is not unique.
Therefore $Q$ must be positive definite.

\subsection*{Exercise 6.11}
Notice that $f'(x)=2ax+b$, $f''(x)=2a$, and that the first Newton's Method iteration is 
$x_1=x_0-f'(x_0)/f''(x_0)$.
Notice that 
\begin{align*}
    f'(x_1)=2a(x_0-(2ax_0+b)/2a)+b=0
\end{align*}
and
\begin{align*}
    f''(x_0)=2a>0.
\end{align*}
Therefore, $x_1$ is a local minimizer.
Since $f$ is quadratic, it is the unique minimizer.

\subsection{Exercise 7.1}
Take $x,y\in\text{conv}(S)$.
Then $x=\sum_{i=1}^{k_x}\lambda^x_is_i$ where $s_i$ are elements of $S$,
$k_x\in\mathbb N$ and $\lambda^x_i$ are nonnegative and sum to $1$.
Do the same for $y$ and set $k=\max\{k_x,k_y\}$.
Also, let $\lambda\in[0,1]$.
Then
\begin{align*}
    \lambda x+(1-\lambda)y = \sum_{i=1}^k(\lambda\lambda^x_i+(1-\lambda)\lambda^y_i)s_i
\end{align*}
where $(\lambda\lambda^x_i+(1-\lambda)\lambda^y_i)$ are nonnegative and
\begin{align*}
    \sum_i\lambda\lambda^x_i+(1-\lambda)\lambda^y_i=
    \lambda\sum_i\lambda^x_i+(1-\lambda)\sum_i\lambda^y_i=1.
\end{align*}
Thus $\lambda x+(1-\lambda y)\in S$ and $S$ is convex.

\subsection*{Exercise 7.2}
(i)
Let $P=\{x\in V\ :\ <a,x>=b\}$ for some $a\in V$, $a\neq 0$ and some real $b$.
Let $x,y\in P$ and $0\leq\lambda\leq 1$.
Then
\begin{align*}
    <a,\lambda x+(1-\lambda)y>=\lambda<a,x>+(1-\lambda)<a,y>=\lambda b + (1-\lambda)b=b.
\end{align*}
Thus $P$ is convex.

(ii)
Let $H=\{x\in\mathbb R^n\ :\ <a,x>\leq b\}$ where again $a\in V$, $a\neq 0$ and some real $b$.
Let $x,y\in H$ and $0\leq\lambda\leq 1$.
Then 
\begin{align*}
    <a,\lambda x+(1-\lambda)y>=\lambda<a,x>+(1-\lambda)<a,y>\leq\lambda b + (1-\lambda)b=b.
\end{align*}
Thus $H$ is convex.

\begin{align*}
       <a,\lambda x+(1-\lambda)y>=\lambda<a,x>+(1-\lambda)<a,y>\leq\lambda b + (1-\lambda)b=b.
\end{align*}

\subsection*{Exercise 7.4}
(i)
Note that
\begin{align*}
    ||x-y||^2&=||(x-p)+(p-y)||^2=\\
    <(x-p)+(p-y),(x-p)+(p-y)>&=
    ||x-p||^2+||p-y||^2+2<x-p,p-y>.
\end{align*}

(ii)
Take an arbitrary $y\neq p$. Then $||p-y||^2> 0$.
Suppose $<x-p,p-y>\geq0$. Then it is clear by (i)
that $||x-y||>||x-p||$.

(iii)
Let $z=\lambda y+(1-\lambda)p$, where $0\leq\lambda\leq1$.
Then by (i) where we use $z$ instead of $y$ we get
\begin{align*}
    ||x-z||^2&=
    ||x-p||^2+||\lambda y-\lambda p||^2+
    <x-p,\lambda p-\lambda y>=\\
    &||x-p||^2+2\lambda<x-p,p-y>+\lambda^2||y-p||^2.
\end{align*}

(iv)
In (7.15), put $\lambda=1$, so $x=y$.
Then we know that
\begin{align*}
    0\leq||x-y||^2-||x-p||^2=2\lambda<x-p,p-y>+2\lambda^2||y-p||^2.
\end{align*}
Dividing by $\lambda$ you get $0\leq2\lambda<x-p,p-y>+2\lambda^2||y-p||^2$.
Take $y=p$, then $0\leq2\lambda<x-p,p-y>$, which clearly implies $0\leq<x-p,p-x>$.

The if statment of the theorem follows by (iv).
The only if statment of the theore follows by (ii).

\subsection*{Exercise 7.8}
Let $x, y\in\mathbb R^n$, with $x\neq y$, and $\lambda\in[0,1]$.
Then
\begin{align*}
    g(\lambda x+(1-\lambda)y)&=
    f(\lambda Ax + (1-\lambda)Ay + b)=
    f(\lambda(Ax+b) + (1-\lambda)(Ay+b))\\
    &\leq\lambda f(Ax+b) + (1-\lambda)f(Ay+b)=
    \lambda g(x) + (1-\lambda)g(y)
\end{align*}
shows that $g$ is convex.

\subsection*{Exercise 7.12}
(i)

Take $X,Y\in PD_n(\mathbb R)$ and $\lambda\in[0,1]$.
Then for every $v\in\mathbb R^n$ we have that
\begin{align*}
    v^T(\lambda X+(1-\lambda)Y)v=
    \lambda(v^TXv)+(1-\lambda)(v^TYv)>0,
\end{align*}
because $X$ and $Y$ are positive definite.

(ii)

(a)
Take $t_1, t_2\in\mathbb R$ and $\lambda\in[0,1]$.
On the one hand, 
\begin{align*}
    \lambda g(t_1) + (1-\lambda)g(t_2) =
    \lambda f(t_1A+(1-t_1)B) + (1-\lambda)f(t_2A+(1-t_2)B).
\end{align*}
On the other, 
\begin{align*}
    g(\lambda t_1 + (1-\lambda)t_2) =&
    f((\lambda t_1+(1-\lambda)t_2)A + (1-\lambda t_1+(1-\lambda)t_2)B)=\\
    &f(\lambda(t_1A+(1-t_1)B)+(1-\lambda)(t_2A+(1-t_2)B)).
\end{align*}
Since $g$ is convex we get
\begin{align*}
    f(\lambda X+(1-\lambda)Y)\leq\lambda f(X)+(1-\lambda)f(Y),
\end{align*}
with $X=t_1A+(1-t_1)B$ and $Y=t_2A+(1-t_2)B$.
Since the choice of $t$ was arbitrary and this holds for any $A,B\in PD_n(\mathbb R)$,
we conclude that $f$ is convex.

(b)
By Proposition $(4.5.7)$, we know that if $A$ is posititve definite, then there exits a nonsingular matrix
$S$ such that $A=S^HS$. Then, $tA+(1-t)B=S^H(tI+(1-t)(S^H)^{-1}BS^{-1})S$,
and so
\begin{align*}
    g(t) = -\log(\text{det}(tA+(1-t)B))=
    -\log(\text{det}(S^H(tI+(1-t)(S^H)^{-1}BS^{-1})S)).
\end{align*}
By the fact that $\text{det}(AB)=\text{det}(A)\text{det}(B)$ and the properties of logarithms,
we obtain
\begin{align*}
    -\log(\text{det}(S^H(tI+(1-t)(S^H)^{-1}BS^{-1})S))=&
    -\log(\text{det}(S^H)) - \log(\text{det}(tI+(1-t)(S^H)^{-1}BS^{-1})) - \log(\text{det}(S))\\
    &-\log(\text{det}(S^H)\text{det}(S)) - \log(\text{det}(tI+(1-t)(S^H)^{-1}BS^{-1}))=\\
    &-\log(\text{det}(A))- \log(\text{det}(tI+(1-t)(S^H)^{-1}BS^{-1})).
\end{align*}

(c)

Since $A,B\in PD_n(\mathbb R)$, then $A^{-1}\in PD_n(\mathbb R)$ and 
\begin{align*}
    &\text{det}((S^H)^{-1}BS^{-1})=
    \text{det}((S^H)^{-1})\text{det}(B)\text{det}(S^{-1})=\\
    &\text{det}(S^{-1})\text{det}((S^H)^{-1})\text{det}(B)=
    \text{det}(A^{-1})\text{det}(B)>0,
\end{align*}
and so $(S^H)^{-1}BS^{-1}$ is full rank.
Now let $\{\lambda_i\}_i$ be the collection of eigenvalues of $((S^H)^{-1}BS^{-1})$ 
and $\{x_i\}_i$ the corresponding collection of eigenvectors. Then for every $i$:
\begin{align*}
    (tI+(1-t)(S^H)^{-1}BS^{-1})x_i=
    tx_i + (1-t)\lambda_ix_i=
    (t+(1-t)\lambda_i)x_i.
\end{align*}
Thus, $\{t + (1-t)\lambda_i\}_i$ are the eigenvalues of $(tI+(1-t)(S^H)^{-1}BS^{-1})$ 
corresponding to the $\{x_i\}_i$, and we can conclude that
\begin{align*}
    -\log(\text{det}(A))- \log(\text{det}(tI+(1-t)(S^H)^{-1}BS^{-1}))=&
    -\log(\text{det}(A))- \log(\Pi_{i=1}^n(t + (1-t)\lambda_i))=\\
    &-\log(\text{det}(A))- \sum_{i=1}^n\log((t + (1-t)\lambda_i)).
\end{align*}

(d)

By using the expression of $g(t)$ in part (c) we can see that
$g'(t)\sum_{i=1}^n(1-\lambda_i)/(t+(1-t)\lambda_i)$ and
$g''(t)=\sum_{i=1}^n(1-\lambda_i)^2/(t+(1-t)\lambda_i)^2$, 
which is clearly nonnegative for all $t\in[0,1]$.

\subsection*{Exercise 7.13}
Suppose $f(x)<M$ for all $x$ for some real $M$ and $f$ is convex and not constant. 
Then, there exist $x,y\in\mathbb R^n$ such that $f(x)\neq f(y)$.
But then the line between $(x,f(x))$ and $(y,f(y))$ intersects $f(\cdot)=M$.
Since $f$ must lie on or above this line, at some point it must cross $f(\cdot)=M$ as well, which is a contraddiction.

\subsection*{Exercise 7.20}
Take $x,y\in\mathbb R^n$, with $x\neq y$, and $\lambda\in[0,1]$.
Since $f$ is convex we have $f(\lambda x+(1-\lambda)y)\leq\lambda f(x)+(1-\lambda)f(y)$.
Since $-f$ is convex, the opposite hold.
Therefore we must have $f(\lambda x+(1-\lambda) y) = \lambda f(x)+(1-\lambda)f(y)$.
Therefore $f$ is affine.

\subsection*{Exercise 7.21}
Let $x^*\in\mathbb R^n$ be a local minimizer of $f$.
Then $f(x^*)\leq f(x)$ for all $x\in\mathcal N_r(x^*)$,
where $\mathcal N_r(x^*)$ is an open ball around $x^*$ of radius $r>0$.
Since $\phi$ is monothonically increasing, $\phi(f(x^*))\leq\phi(f(x))$ for all $x\in\mathcal N_r(x^*)$.
Thus, $x^*$ is a local minimizer of $\phi\circ f$.
Now let $x^*$ be a local minimizer of $\phi\circ f$.
Then $\phi(f(x^*))\leq\phi(f(x))$ for all $x\in\mathcal N_r(x^*)$,
and since $\phi$ is monothonically increasing, this implies that
$f(x^*)\leq f(x)$ for all $x\in\mathcal N_r(x^*)$.
Thus, $x^*$ is a local minimizer of $f$.


\end{document}

